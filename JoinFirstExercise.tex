% Options for packages loaded elsewhere
\PassOptionsToPackage{unicode}{hyperref}
\PassOptionsToPackage{hyphens}{url}
%
\documentclass[
]{article}
\usepackage{lmodern}
\usepackage{amssymb,amsmath}
\usepackage{ifxetex,ifluatex}
\ifnum 0\ifxetex 1\fi\ifluatex 1\fi=0 % if pdftex
  \usepackage[T1]{fontenc}
  \usepackage[utf8]{inputenc}
  \usepackage{textcomp} % provide euro and other symbols
\else % if luatex or xetex
  \usepackage{unicode-math}
  \defaultfontfeatures{Scale=MatchLowercase}
  \defaultfontfeatures[\rmfamily]{Ligatures=TeX,Scale=1}
\fi
% Use upquote if available, for straight quotes in verbatim environments
\IfFileExists{upquote.sty}{\usepackage{upquote}}{}
\IfFileExists{microtype.sty}{% use microtype if available
  \usepackage[]{microtype}
  \UseMicrotypeSet[protrusion]{basicmath} % disable protrusion for tt fonts
}{}
\makeatletter
\@ifundefined{KOMAClassName}{% if non-KOMA class
  \IfFileExists{parskip.sty}{%
    \usepackage{parskip}
  }{% else
    \setlength{\parindent}{0pt}
    \setlength{\parskip}{6pt plus 2pt minus 1pt}}
}{% if KOMA class
  \KOMAoptions{parskip=half}}
\makeatother
\usepackage{xcolor}
\IfFileExists{xurl.sty}{\usepackage{xurl}}{} % add URL line breaks if available
\IfFileExists{bookmark.sty}{\usepackage{bookmark}}{\usepackage{hyperref}}
\hypersetup{
  pdftitle={Join First Exercises},
  pdfauthor={Ebru Gecici},
  hidelinks,
  pdfcreator={LaTeX via pandoc}}
\urlstyle{same} % disable monospaced font for URLs
\usepackage[margin=1in]{geometry}
\usepackage{color}
\usepackage{fancyvrb}
\newcommand{\VerbBar}{|}
\newcommand{\VERB}{\Verb[commandchars=\\\{\}]}
\DefineVerbatimEnvironment{Highlighting}{Verbatim}{commandchars=\\\{\}}
% Add ',fontsize=\small' for more characters per line
\usepackage{framed}
\definecolor{shadecolor}{RGB}{248,248,248}
\newenvironment{Shaded}{\begin{snugshade}}{\end{snugshade}}
\newcommand{\AlertTok}[1]{\textcolor[rgb]{0.94,0.16,0.16}{#1}}
\newcommand{\AnnotationTok}[1]{\textcolor[rgb]{0.56,0.35,0.01}{\textbf{\textit{#1}}}}
\newcommand{\AttributeTok}[1]{\textcolor[rgb]{0.77,0.63,0.00}{#1}}
\newcommand{\BaseNTok}[1]{\textcolor[rgb]{0.00,0.00,0.81}{#1}}
\newcommand{\BuiltInTok}[1]{#1}
\newcommand{\CharTok}[1]{\textcolor[rgb]{0.31,0.60,0.02}{#1}}
\newcommand{\CommentTok}[1]{\textcolor[rgb]{0.56,0.35,0.01}{\textit{#1}}}
\newcommand{\CommentVarTok}[1]{\textcolor[rgb]{0.56,0.35,0.01}{\textbf{\textit{#1}}}}
\newcommand{\ConstantTok}[1]{\textcolor[rgb]{0.00,0.00,0.00}{#1}}
\newcommand{\ControlFlowTok}[1]{\textcolor[rgb]{0.13,0.29,0.53}{\textbf{#1}}}
\newcommand{\DataTypeTok}[1]{\textcolor[rgb]{0.13,0.29,0.53}{#1}}
\newcommand{\DecValTok}[1]{\textcolor[rgb]{0.00,0.00,0.81}{#1}}
\newcommand{\DocumentationTok}[1]{\textcolor[rgb]{0.56,0.35,0.01}{\textbf{\textit{#1}}}}
\newcommand{\ErrorTok}[1]{\textcolor[rgb]{0.64,0.00,0.00}{\textbf{#1}}}
\newcommand{\ExtensionTok}[1]{#1}
\newcommand{\FloatTok}[1]{\textcolor[rgb]{0.00,0.00,0.81}{#1}}
\newcommand{\FunctionTok}[1]{\textcolor[rgb]{0.00,0.00,0.00}{#1}}
\newcommand{\ImportTok}[1]{#1}
\newcommand{\InformationTok}[1]{\textcolor[rgb]{0.56,0.35,0.01}{\textbf{\textit{#1}}}}
\newcommand{\KeywordTok}[1]{\textcolor[rgb]{0.13,0.29,0.53}{\textbf{#1}}}
\newcommand{\NormalTok}[1]{#1}
\newcommand{\OperatorTok}[1]{\textcolor[rgb]{0.81,0.36,0.00}{\textbf{#1}}}
\newcommand{\OtherTok}[1]{\textcolor[rgb]{0.56,0.35,0.01}{#1}}
\newcommand{\PreprocessorTok}[1]{\textcolor[rgb]{0.56,0.35,0.01}{\textit{#1}}}
\newcommand{\RegionMarkerTok}[1]{#1}
\newcommand{\SpecialCharTok}[1]{\textcolor[rgb]{0.00,0.00,0.00}{#1}}
\newcommand{\SpecialStringTok}[1]{\textcolor[rgb]{0.31,0.60,0.02}{#1}}
\newcommand{\StringTok}[1]{\textcolor[rgb]{0.31,0.60,0.02}{#1}}
\newcommand{\VariableTok}[1]{\textcolor[rgb]{0.00,0.00,0.00}{#1}}
\newcommand{\VerbatimStringTok}[1]{\textcolor[rgb]{0.31,0.60,0.02}{#1}}
\newcommand{\WarningTok}[1]{\textcolor[rgb]{0.56,0.35,0.01}{\textbf{\textit{#1}}}}
\usepackage{graphicx}
\makeatletter
\def\maxwidth{\ifdim\Gin@nat@width>\linewidth\linewidth\else\Gin@nat@width\fi}
\def\maxheight{\ifdim\Gin@nat@height>\textheight\textheight\else\Gin@nat@height\fi}
\makeatother
% Scale images if necessary, so that they will not overflow the page
% margins by default, and it is still possible to overwrite the defaults
% using explicit options in \includegraphics[width, height, ...]{}
\setkeys{Gin}{width=\maxwidth,height=\maxheight,keepaspectratio}
% Set default figure placement to htbp
\makeatletter
\def\fps@figure{htbp}
\makeatother
\setlength{\emergencystretch}{3em} % prevent overfull lines
\providecommand{\tightlist}{%
  \setlength{\itemsep}{0pt}\setlength{\parskip}{0pt}}
\setcounter{secnumdepth}{-\maxdimen} % remove section numbering
\ifluatex
  \usepackage{selnolig}  % disable illegal ligatures
\fi

\title{Join First Exercises}
\author{Ebru Gecici}
\date{25 08 2020}

\begin{document}
\maketitle

{
\setcounter{tocdepth}{3}
\tableofcontents
}
To get information about the \texttt{join()} function, which belongs to
\texttt{tidyverse} package, this exercises are made by using
\href{https://stat545.com/join-cheatsheet.html}{tutorial}.

There are some join functions:

\begin{itemize}
\item
  \texttt{left\_join()} and \texttt{right\_join()}
\item
  \texttt{inner\_join()}
\item
  \texttt{semi\_join()}
\item
  \texttt{full\_join()}
\item
  \texttt{anti\_join()}
\end{itemize}

\begin{Shaded}
\begin{Highlighting}[]
\KeywordTok{library}\NormalTok{(tidyverse) }\CommentTok{\#\# dplyr provides the join functions}
\KeywordTok{library}\NormalTok{(tinytex)}

\NormalTok{superheroes \textless{}{-}}\StringTok{ }\NormalTok{tibble}\OperatorTok{::}\KeywordTok{tribble}\NormalTok{(}
       \OperatorTok{\textasciitilde{}}\NormalTok{name, }\OperatorTok{\textasciitilde{}}\NormalTok{alignment,  }\OperatorTok{\textasciitilde{}}\NormalTok{gender,          }\OperatorTok{\textasciitilde{}}\NormalTok{publisher,}
   \StringTok{"Magneto"}\NormalTok{,      }\StringTok{"bad"}\NormalTok{,   }\StringTok{"male"}\NormalTok{,            }\StringTok{"Marvel"}\NormalTok{,}
     \StringTok{"Storm"}\NormalTok{,     }\StringTok{"good"}\NormalTok{, }\StringTok{"female"}\NormalTok{,            }\StringTok{"Marvel"}\NormalTok{,}
  \StringTok{"Mystique"}\NormalTok{,      }\StringTok{"bad"}\NormalTok{, }\StringTok{"female"}\NormalTok{,            }\StringTok{"Marvel"}\NormalTok{,}
    \StringTok{"Batman"}\NormalTok{,     }\StringTok{"good"}\NormalTok{,   }\StringTok{"male"}\NormalTok{,                }\StringTok{"DC"}\NormalTok{,}
     \StringTok{"Joker"}\NormalTok{,      }\StringTok{"bad"}\NormalTok{,   }\StringTok{"male"}\NormalTok{,                }\StringTok{"DC"}\NormalTok{,}
  \StringTok{"Catwoman"}\NormalTok{,      }\StringTok{"bad"}\NormalTok{, }\StringTok{"female"}\NormalTok{,                }\StringTok{"DC"}\NormalTok{,}
   \StringTok{"Hellboy"}\NormalTok{,     }\StringTok{"good"}\NormalTok{,   }\StringTok{"male"}\NormalTok{, }\StringTok{"Dark Horse Comics"}
\NormalTok{  )}

\NormalTok{publishers \textless{}{-}}\StringTok{ }\NormalTok{tibble}\OperatorTok{::}\KeywordTok{tribble}\NormalTok{(}
  \OperatorTok{\textasciitilde{}}\NormalTok{publisher, }\OperatorTok{\textasciitilde{}}\NormalTok{yr\_founded,}
        \StringTok{"DC"}\NormalTok{,       1934L,}
    \StringTok{"Marvel"}\NormalTok{,       1939L,}
     \StringTok{"Image"}\NormalTok{,       1992L}
\NormalTok{  )}

\NormalTok{superheroes}
\end{Highlighting}
\end{Shaded}

\begin{verbatim}
## # A tibble: 7 x 4
##   name     alignment gender publisher        
##   <chr>    <chr>     <chr>  <chr>            
## 1 Magneto  bad       male   Marvel           
## 2 Storm    good      female Marvel           
## 3 Mystique bad       female Marvel           
## 4 Batman   good      male   DC               
## 5 Joker    bad       male   DC               
## 6 Catwoman bad       female DC               
## 7 Hellboy  good      male   Dark Horse Comics
\end{verbatim}

\begin{Shaded}
\begin{Highlighting}[]
\NormalTok{publishers}
\end{Highlighting}
\end{Shaded}

\begin{verbatim}
## # A tibble: 3 x 2
##   publisher yr_founded
##   <chr>          <int>
## 1 DC              1934
## 2 Marvel          1939
## 3 Image           1992
\end{verbatim}

There are two different data sets, i.e., superheroes and publishers. By
using these two data set, we can obtain a data set which is the
combination of the superheroes and publishers. To make this merge
process, we need to use common data of data sets that is
\textbf{publisher}.

In these examples:

x = superheroes, and y=publishers

\hypertarget{left-join}{%
\subsubsection{1. Left Join}\label{left-join}}

In the left join, the all rows of X are preserved and the only relevant
columns of y is used. If there is no matching of the x values on the y
table, this rows returns \texttt{NA} values.

\texttt{Left\ Join:\ All\ rows\ of\ X\ are\ preserved,\ only\ relevant\ rows\ Y\ and\ multiply\ rows\ if\ there\ are\ matching.}

\textbf{Note that}, the logic behind the right join is same as the left
join. For this reason, to prevent confusion, only one of them can be
used.

\begin{Shaded}
\begin{Highlighting}[]
\KeywordTok{left\_join}\NormalTok{(superheroes, publishers, }\DataTypeTok{by=}\StringTok{"publisher"}\NormalTok{)}
\end{Highlighting}
\end{Shaded}

\begin{verbatim}
## # A tibble: 7 x 5
##   name     alignment gender publisher         yr_founded
##   <chr>    <chr>     <chr>  <chr>                  <int>
## 1 Magneto  bad       male   Marvel                  1939
## 2 Storm    good      female Marvel                  1939
## 3 Mystique bad       female Marvel                  1939
## 4 Batman   good      male   DC                      1934
## 5 Joker    bad       male   DC                      1934
## 6 Catwoman bad       female DC                      1934
## 7 Hellboy  good      male   Dark Horse Comics         NA
\end{verbatim}

\hypertarget{inner-join}{%
\subsubsection{2. Inner Join}\label{inner-join}}

\texttt{Inner\ Join:\ Only\ rows\ with\ common\ value\ are\ returned\ and\ rws\ rae\ multiplied\ if\ there\ are\ multiple\ matchings}

\begin{Shaded}
\begin{Highlighting}[]
\KeywordTok{inner\_join}\NormalTok{(superheroes, publishers, }\DataTypeTok{by=}\StringTok{"publisher"}\NormalTok{)}
\end{Highlighting}
\end{Shaded}

\begin{verbatim}
## # A tibble: 6 x 5
##   name     alignment gender publisher yr_founded
##   <chr>    <chr>     <chr>  <chr>          <int>
## 1 Magneto  bad       male   Marvel          1939
## 2 Storm    good      female Marvel          1939
## 3 Mystique bad       female Marvel          1939
## 4 Batman   good      male   DC              1934
## 5 Joker    bad       male   DC              1934
## 6 Catwoman bad       female DC              1934
\end{verbatim}

Inner join is similar with the left join, but there is difference in the
NA values. If there is an NA value in the rows, inner join does not
return this rows in the merged data frame.

\hypertarget{semi-join}{%
\subsubsection{3. Semi Join}\label{semi-join}}

\texttt{Semi\ Join:\ Very\ similar\ to\ inner\ join\ but\ without\ columns\ from\ Y}

\begin{Shaded}
\begin{Highlighting}[]
\KeywordTok{semi\_join}\NormalTok{(superheroes, publishers, }\DataTypeTok{by=}\StringTok{"publisher"}\NormalTok{)}
\end{Highlighting}
\end{Shaded}

\begin{verbatim}
## # A tibble: 6 x 4
##   name     alignment gender publisher
##   <chr>    <chr>     <chr>  <chr>    
## 1 Magneto  bad       male   Marvel   
## 2 Storm    good      female Marvel   
## 3 Mystique bad       female Marvel   
## 4 Batman   good      male   DC       
## 5 Joker    bad       male   DC       
## 6 Catwoman bad       female DC
\end{verbatim}

This function provide the merged of the data sets but, it does not
include the NA value of the X data set, it removes this data and
moreover,semi join function does not show column of the y, publisher,
data set.

\hypertarget{full-join}{%
\subsubsection{4. Full Join}\label{full-join}}

\texttt{Full\ join\ returns\ all\ rows\ and\ columns\ from\ both\ X\ and\ Y\ and\ both\ multiple\ mathcings\ and\ compensates\ for\ missing}

\begin{Shaded}
\begin{Highlighting}[]
\KeywordTok{full\_join}\NormalTok{(superheroes, publishers, }\DataTypeTok{by =} \StringTok{"publisher"}\NormalTok{)}
\end{Highlighting}
\end{Shaded}

\begin{verbatim}
## # A tibble: 8 x 5
##   name     alignment gender publisher         yr_founded
##   <chr>    <chr>     <chr>  <chr>                  <int>
## 1 Magneto  bad       male   Marvel                  1939
## 2 Storm    good      female Marvel                  1939
## 3 Mystique bad       female Marvel                  1939
## 4 Batman   good      male   DC                      1934
## 5 Joker    bad       male   DC                      1934
## 6 Catwoman bad       female DC                      1934
## 7 Hellboy  good      male   Dark Horse Comics         NA
## 8 <NA>     <NA>      <NA>   Image                   1992
\end{verbatim}

Full join, merged all data of the superheroes and publishers. For this
reason, we can see more NA values in the merged data frame.

\hypertarget{anti-join}{%
\subsubsection{5. Anti Join}\label{anti-join}}

\texttt{Anti\ join\ returns\ all\ rows\ from\ X\ which\ do\ not\ have\ information\ (based\ on\ key\ column)\ in\ Y\ and\ returns\ only\ columns\ from\ X.}

\begin{Shaded}
\begin{Highlighting}[]
\KeywordTok{anti\_join}\NormalTok{(superheroes, publishers, }\DataTypeTok{by =} \StringTok{"publisher"}\NormalTok{)}
\end{Highlighting}
\end{Shaded}

\begin{verbatim}
## # A tibble: 1 x 4
##   name    alignment gender publisher        
##   <chr>   <chr>     <chr>  <chr>            
## 1 Hellboy good      male   Dark Horse Comics
\end{verbatim}

By using anti join function, we can obtain values, which have no
matching in the second data frame.. In this example, Dark Horse Comics
does not take part in the publishers data set, it is not macth.

\begin{Shaded}
\begin{Highlighting}[]
\KeywordTok{anti\_join}\NormalTok{(publishers, superheroes, }\DataTypeTok{by=}\StringTok{"publisher"}\NormalTok{)}
\end{Highlighting}
\end{Shaded}

\begin{verbatim}
## # A tibble: 1 x 2
##   publisher yr_founded
##   <chr>          <int>
## 1 Image           1992
\end{verbatim}

In the second anti join function, we use publishers as x values. For
this reason, the function shows that the value which does not take part
in the y data set, i.e., superheroes in this trial.

\hypertarget{reference}{%
\subsubsection{6. Reference}\label{reference}}

This examples are made according to the information in the IE48A class
and also the data used from the study of
\href{https://stat545.com/join-cheatsheet.html}{Stat545}.

\end{document}
